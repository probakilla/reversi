\documentclass[10pt,a4paper]{article}

\usepackage[colorlinks=true,urlcolor=blue,linkcolor=blue]{hyperref}
\usepackage[utf8]{inputenc}
\usepackage[T1]{fontenc}
\usepackage[french]{babel}
\usepackage{fancyvrb}

\usepackage{float}
\usepackage{graphicx}

\title{Cahier des besoins - Reversi}
\author{Guillaume CHUPIN, Benoit FAGET, Alexis PICHON, Julien PILLEUX}

\begin {document}
\maketitle
\newpage
\tableofcontents
\newpage

\section{Introduction}

Notre projet consiste en l'élaboration d'un joueur de Reversi complet développé en suivant certaines règles afin de pouvoir être confronté à d'autre implémentations au cours d'un tournoi prévu qui confrontera notre progamme à celui d'une seconde équipe de développement ayant reçu le même sujet.\\

Nous avons eu le choix entre utiliser le langage de programmation C ou le C++, nous avons décidé de prendre le C++. Notre programme appliquera les algorithmes les plus fréquents pour un joueur de Reversi et utilisera au mieux les bitboards. L'interface utilisateur sera purement en mode texte dans un bash, on se concentrera plutôt sur le développement d'heuristiques propres au jeu. Le projet sera essentiellement dirigé vers l'utilisation de techniques d'exploration d'arbre classiques telles que Minimax-ab, Negamax, Negascout, MTD(f), Monte-Carlo Tree Search (MCTS), etc. Il nous reste à définir laquelle développer en première, et si nous aurons le temps d'en faire d'autres.

\section{Description et analyse de l'existant}

Le jeu de Reversi consiste en un jeu de pions opposant deux adversaires dont le but est soit d'éliminer tous les pions adverses, soit de terminer avec le plus grand nombre de pions à la fin de la partie. La partie de termine lorque plus aucun mouvement n'est possible pour chacun des deux joueurs. Un joueur ne peut placer un pion que si ce dernier permet la capture d'au moins un pion adverse. Pour capturer les pions adversaires, il faut que ceux-ci se retrouve entre un de vos pions déjà présent et celui que vous vous apprêtez à poser sur le plateau. La capture de pions peut s'effectuer dans les huit directions à la fois.

\subsection{Stratégies}

Différentes stratégies autour du jeu existent. En effet, certaines positions sont meilleures que d'autres, car offrant moins de possibilités à l'adversaire de la reprendre, si ce n'est aucune. Par exemple, les quatre coins du plateau font parties de ces positions imprenables. De plus, une fois ces positions prises, les positions adjacentes deviennent également imprenables, et ainsi de suite (voir FIGURE \ref{bord_stable}).

\begin{figure}[H]
\centering
\includegraphics[scale=0.8]{images/bord_stable.png}
\label{bord_stable}
\caption{Quatre coins du plateau}
\end{figure}

D'autres positions imprenables sont celles qui se retrouvent entre des pions adverses, car ce dernier ne pourra pas non plus les reprendre. Par exemple, dans la FIGURE \ref{exemple_insertion}, on peut voir que le joueur noir (représenté par un 'X') peut effectuer une insertion en jouant en 'A3', le joueur blanc ne pourra pas reprendre ce pion. Au tour suivant, quelque soit le coup du joueur blanc, le joueur noir pourra jouer en 'A1' et ainsi avoir un des coins si désirés, lui permetant ainsi de gagner en stabilité en possédant maintenant trois pions indétronables.

Une autre stratégie reviendrait à diminuer le nombre de coups de l'adversaire tout en augmentant son propre nombre de coups, pour cela on va privilégier les coups rapportant le moins de pions, mais tout en faisant attention à ce que l'adversaire ne puisse pas nous enlever tous nos pions. Réduire le nombre de coups de l'adversaire n'est pas tout, il faut aussi tendre à ne lui laisser la possibilité de ne jouer que des mauvais coups.\\

Comme nous pouvons l'observer à travers les différentes stratégies, nous aurons besoin d'explorer les arbres des coups possibles, mais la complexité d'une telle démarche devient rapidement gigantesque, nous empêchant de parcourir en un temps raisonnable l'arbre en entier. Fort heureusement, nous n'aurons pas besoin de le parcourir en entier, car nous pourrons déterminer que certaines branches de l'arbre ne valent pas le coup d'être explorées.

\subsection{Programmes}

À l'heure actuelle, le programme qui est le champion à ce jeu se nomme Iago. Il se base sur les stratégies énoncées précédemment, mais de part la contrainte technique de temps dûe aux tournois, ils ont dû utiliser certaines astuces. Ainsi, le programme calcule pour chaque coup le temps qu'il peut passer sur celui-ci. Ensuite, durant le temps alloué pour chaque coup, il recherche dans son arbre jusqu'à une profondeur de 7, qu'ils ont jugée comme étant la profondeur donnant le meilleur résultat dans un temps restreint. Pour réduire son temps de réponse, ils ont aussi amélioré les fonctions d'évaluation, mais ils ont surtout réduit le nombre de noeuds que le programme devra examiner.
\newpage

\section{Besoins fonctionnels}

\textbf{Lancer une nouvelle partie de Reversi entre deux utilisateurs :}\\
\begin{itemize}
\item Lancer le programme sans option dans une console de type bash ;
\item Afficher un message de bienvenue et des explications sur le déroulement de la partie (voir \ref{board}) ;\\
\item Initialiser la grille de jeu avec ses dimensions par défaut (voir \ref{board}) ;
\item Déterminer les mouvements possibles pour le joueur en cours ;
\item Mettre à jour la grille de jeu avec les mouvements possibles ;
\item Afficher la grille de jeu ;\\
\item (*) Proposer à l'utilisateur de saisir son mouvement ou de quitter la partie (voir \ref{board}) ;
\item Vérifier que la saisie est valide, retourner à l'étape précédente si ce n'est pas le cas ;\\
\end{itemize}
Si l'utilisateur décide d'un mouvement à réaliser :
\begin{itemize}
\item Vérifier que le mouvement est valide, retourner à l'étape précédente si ce n'est pas le cas ;
\item Mettre à jour la grille de jeu avec le mouvement fraîchement réalisé ;
\item Terminer le tour et changer de joueur ;
\item Déterminer les mouvements possibles pour le joueur en cours;
\item Mettre à jour la grille de jeu avec les mouvements possibles ;
\item Afficher la grille de jeu ;
\item Si la partie se termine, afficher le vainqueur et quitter le programme.
Sinon, retourner à l'étape (*) ;\\
\end{itemize}
Si l'utilisateur décide de quitter la partie :
\begin{itemize}
\item Proposer à l'utilisateur de sauvegarder l'état de la partie en cours avant de quitter (voir \ref{save}) ;
\item Vérifier que la saisie est valide, retourner à l'étape précédente si ce n'est pas le cas ;
\item Si souhaité, sauvegarder l'état de la partie en cours dans un fichier (voir \ref{save}) ;
\item Quitter le programme ;
\end{itemize}
\newpage

\textbf{Lancer une nouvelle partie de Reversi entre l'utilisateur et l'ordinateur :}\\
\begin{itemize}
\item Lancer le programme avec l'option -b ou -w dans une console de type bash (voir \ref{options}) ;
\item Afficher un message de bienvenue et des explications sur le déroulement de la partie (voir \ref{board}) ;\\
\item Initialiser la grille de jeu avec ses dimensions par défaut (voir \ref{board}) ;
\item Déterminer les mouvements possibles pour le joueur en cours ;
\item Mettre à jour la grille de jeu avec les mouvements possibles ;
\item Afficher la grille de jeu ;\\
\end{itemize}
(*) Si c'est au tour de l'utilisateur :
\begin{itemize}
\item Proposer à l'utilisateur de saisir son mouvement ou de quitter la partie (voir \ref{board}) ;
\item Vérifier que la saisie est valide, retourner à l'étape précédente si ce n'est pas le cas ;
\end{itemize}
Si l'utilisateur décide d'un mouvement à réaliser :
\begin{itemize}
\item Vérifier que le mouvement est valide, retourner à l'étape précédente si ce n'est pas le cas ;
\item Mettre à jour la grille de jeu avec le mouvement fraîchement réalisé ;
\item Terminer le tour et changer de joueur ;
\item Déterminer les mouvements possibles pour le joueur en cours ;
\item Mettre à jour la grille de jeu avec les mouvements possibles ;
\item Afficher la grille de jeu ;
\item Si la partie se termine, afficher le vainqueur et quitter le programme.
Sinon, retourner à l'étape (*) ;\\
\end{itemize}
Si l'utilisateur décide de quitter la partie :
\begin{itemize}
\item Proposer à l'utilisateur de sauvegarder l'état de la partie en cours avant de quitter (voir \ref{save}) ;
\item Vérifier que la saisie est valide, retourner à l'étape précédente si ce n'est pas le cas ;
\item Si souhaité, sauvegarder l'état de la partie en cours dans un fichier (voir \ref{save}) ;
\item Quitter le programme ;\\
\end{itemize}
(*) Si c'est au tour de l'ordinateur :
\begin{itemize}
\item Calculer un mouvement à l'aide de la technique d'exploration d'arbre classique choisie ;
\item Mettre à jour la grille de jeu avec le mouvement fraîchement calculé ;
\item Terminer le tour et changer de joueur ;
\item Déterminer les mouvements possibles pour le joueur en cours ;
\item Mettre à jour la grille de jeu avec les mouvements possibles ;
\item Afficher la grille de jeu ;
\item Si la partie se termine, afficher le vainqueur et quitter le programme.
Sinon, retourner à l'étape (*) ;\\
\end{itemize}
\newpage

\textbf{Lancer une nouvelle partie de Reversi entre deux ordinateurs :}\\
\begin{itemize}
\item Lancer le programme avec l'option -a dans une console de type bash (voir \ref{options}) ;
\item Afficher un message de bienvenue et des explications sur le déroulement de la partie (voir \ref{board}) ;\\
\item Initialiser la grille de jeu avec ses dimensions par défaut (voir \ref{board}) ;
\item Déterminer les mouvements possibles pour le joueur en cours ;
\item Mettre à jour la grille de jeu avec les mouvements possibles ;
\item Afficher la grille de jeu ;\\
\item (*) Calculer un mouvement à l'aide de la technique d'exploration d'arbre classique choisie ;
\item Mettre à jour la grille de jeu avec le mouvement fraîchement calculé ;
\item Terminer le tour et changer de joueur ;
\item Déterminer les mouvements possibles pour le joueur en cours ;
\item Mettre à jour la grille de jeu avec les mouvements possibles ;
\item Afficher la grille de jeu ;
\item Si la partie se termine, afficher le vainqueur et quitter le programme.
Sinon, retourner à l'étape (*) ;\\
\end{itemize}

\textbf{Répondre par un mouvement à un état de partie :}
\begin{itemize}
\item Lancer le programme avec l'option -c suivie du nom de fichier contenant l'état de partie dans une console de type bash (voir \ref{options}) ;\\
\item Récupérer le joueur en cours et la grille de jeu à partir des données lues dans le fichier ;
\item Déterminer les mouvements possibles pour le joueur en cours ;
\item Mettre à jour la grille de jeu avec les mouvements possibles ;
\item Afficher la grille de jeu ;\\
\item Calculer un mouvement à l'aide de la technique d'exploration d'arbre classique choisie ;
\item Mettre à jour la grille de jeu avec le mouvement fraîchement calculé ;
\item Terminer le tour et changer de joueur ;
\item Mettre à jour la grille de jeu avec les mouvements possibles ;
\item Afficher la grille de jeu ;
\item Si la partie se termine, afficher le vainqueur et quitter le programme.
Sinon, proposer à l'utilisateur de sauvegarder l'état de la partie en cours avant de quitter (voir \ref{save}) ;
\item Vérifier que la saisie est valide, retourner à l'étape précédente si ce n'est pas le cas ;
\item Si souhaité, sauvegarder l'état de la partie en cours dans un fichier (voir \ref{save}) ;
\item Quitter le programme ;\\
\end{itemize}
\newpage

\textbf{Lancer une partie en modifiant les dimensions de la grille de jeu :}
\begin{itemize}
\item Lancer le programme avec l'option -s suivie d'un entier compris entre 1 et 5 dans une console de type bash (voir \ref{options}) ;
\item Voir les descriptions de parties précédentes ;\\
\end{itemize}

\textbf{Afficher la liste des options disponibles et leurs descriptions :}
\begin{itemize}
\item Lancer le programme avec l'option -h dans une console de type bash (voir \ref{options}) ;
\item Afficher la liste de options disponibles et leurs descriptions ;\\
\end{itemize}

\textbf{Lancer une partie en mode verbeux :}
\begin{itemize}
\item Lancer le programme avec les options voulues ainsi que l'option -v dans une console de type bash (voir \ref{options}) ;
\item Voir les descriptions de parties précédentes, toutes les opérations effectuées seront affichées explicitement ;\\
\end{itemize}

\textbf{Afficher la version du programme :}
\begin{itemize}
\item Lancer le programme avec l'option -V dans une console de type bash (voir \ref{options}) ;
\item Afficher la version du programme ;
\end{itemize}

% Début nouvelle partie post TD1
\begin {itemize}
\item  L'entrée de l'utilisateur est insensible à la casse.
\item  L'utilisateur peut choisir un coup à jouer en entrant les coordonnées de la case où il désire jouer, par exemple il peut choisir la case \verb!a2!.
\item  L'utilisateur peut sauvegarder sa partie à tout moment en entrant le caractère 'q'. Le fichier de sortie de la sauvegarde est un fichier \verb!ASCII! contenant l'état acutel du plateau plus le tour du prochain joueur (voir Fig. \ref{fig:exemple_save}).
  \begin{figure}[H]    
    \centering
    \begin{BVerbatim}
      O
      _ _ _ _ _ _
      _ _ _ _ _ _
      _ _ X O _ _
      _ _ O X _ _
      _ _ _ _ _ _
      _ _ _ _ _ _ 
    \end{BVerbatim}
    \caption {Exemple de fichier de sauvegarde.\label{fig:exemple_save}}
    \end{figure}

\item  Le programme doit contenir un mode contest peremttant de faire s'affronter deux intelligences artificielles. Ce mode de jeu prend en entrée un fichier de sauvegarde obtenu lorsque l'on quitte le programme.
\item  La taille du plateau de jeu est modifiable grâce grâce à une option de lancement. Lors du lancement -s ou --size suivit d'une valeur entre 1 et 5 ce qui fait varier la taille de la grille de 2 à 10 cases en ne prenant que des valeur paires (Voir Fig. \ref{fig:exemple_taille}). Par défaut, la grille affichée sera de dimention \verb!8x8!.
  \begin{figure}[H]    
    \centering
    \begin{BVerbatim}
        1 2 3 4 
      A _ _ _ _
      B _ _ _ _
      C _ _ _ _
      D _ _ _ _
      
    \end{BVerbatim}
    \caption {Exemple de grille générée avec l'option \textsc{-s2}.\label{fig:exemple_taille}}
  \end{figure}
\item  Il est possible de décider quel rôle donner à l'ia lors du lancement du programme.
  \begin{description}
  \item [-w, --white] Pour indiquer que l'ia jouera les pions blancs.
  \item [-b, --black] Pour indiquer que l'ia jouera les pions noirs.
  \item [-a, --all] Pour indiquer que l'ia jouera les deux camps.
  \end{description}
\item  La version du programme sera affchable avec l'option de lancement -V.
\item  Le mode verbose de la partie sera activable au lancement du programme avec l'option -v.
\item  Une aide est disponible pour l'utilisateur grâce à l'option -h ou --help. Elle liste l'ensemble des otpions du programme ainsi que leur description.
\end{itemize}
% Fin nouvelle partie post TD1.

\section{Besoins non fonctionnels}

\begin{itemize} 
\item Rapidité d'exécution : chaque coup doit être exécuté en moins d'une seconde.
\item Portabilité : le programme doit pouvoir s'exécuter indépendamment de la machine, du système d'exploitation et de ses configurations.
\item Confiance : les règles doivent être respectées et les fichiers de sauvegarde doivent suivre le format et la forme attendue (voir \ref{save}).
\end{itemize}

\begin{figure}[H]
\centering
\includegraphics[scale=0.5]{images/use_case.png}
\label{use_case}
\caption{Diagramme de cas d'utilisation}
\end{figure}

\subsection{Diagrammes de séquence}

\subsection{Diagramme de Gantt}

\section{Bibliographie}

\end{document}
